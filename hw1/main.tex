\documentclass{article}
\usepackage{type1cm}
\usepackage{float}
\usepackage{pythonhighlight}
\usepackage{fontspec}   %加這個就可以設定字體
\usepackage{xeCJK}      %讓中英文字體分開設置
\setCJKmainfont{標楷體} %設定中文為系統上的字型,而英文不去更動,使用原TeX字型
\XeTeXlinebreaklocale "zh"             %這兩行一定要加,中文才能自動換行
\XeTeXlinebreakskip = 0pt plus 1pt     %這兩行一定要加,中文才能自動換行

\title{ 資訊安全與密碼學---第一次作業 \\  \begin{large} AES256-ECB \\ AES256-CBC \\ AES256-CTR \\ RSA2048 \\ SHA512 \\ \end{large} }
\author{\\4103056005 \\ 資工四 \\ 林佑儒}
\date{}

\begin{document}
\maketitle
\newpage
\section{作業目標}
\begin{enumerate}
    \item 產生出一個大小為 512MB+7byte 的隨機檔案,目的在於使其在作加密時需要做 padding.
    \item 使用 Python 的 PyCrypto 套件來做\\AES256-ECB、AES256-CBC、AES256-CTR、RSA2048、SHA512
	\item 使用 Python 的 Cryptography 套件來做\\AES256-ECB、AES256-CBC、AES256-CTR、RSA2048、SHA512
	\item 加密時,若遇到需要做 padding 的情況,需依照 PKCS 的 padding 規範
	\item 測量兩套件各項功能所需的時間並比較之
\end{enumerate}

\section{系統環境}
\quad 本次作業皆是在macOS High Sierra + python2.7的環境下測試執行
\section{安裝套件}
\begin{enumerate}
    \item sudo pip install pycrypto
    \item sudo pip install cryptography
\end{enumerate}
\newpage
\section{PyCrypto}
\subsection{AES256-ECB}
\begin{python}
# PyCrypto_AES256_ECB
#======================================================================
from Crypto.Cipher import AES
import os
import time
#======================================================================
def start():
    global key
    key = os.urandom(32)                    #AES 256 bits = 32 bytes
    file = open("random.txt","r")
    global text                             
    text = file.read()                      #text will be the plaintext
    file.close()
#======================================================================
def en_AES_ECB(key,text):
    padString = ''
    temp = 16-len(text)%16            #counting the padding number
    for i in range(0,temp):           #if surplus 7, then add 9 bytes 9
        padString = padString+chr(temp)
    text = text+padString
    
    cipher = AES.new(key)
    global encrypted 
    encrypted = cipher.encrypt(text)                           #encrypt
#======================================================================
def de_AES_ECB(key,encrypted):
    cipher = AES.new(key)
    decrypted = cipher.decrypt(encrypted)                      #decrypt
    temp = len(decrypted)                                    #unpadding
    temp = temp-int(decrypted[-1].encode('hex'),16)
    decrypted = decrypted[:temp]
#======================================================================
start()
#AES256EBC
print "AES_256_ECB encode:",
startTime = time.time()
en_AES_ECB(key,text)
print time.time()-startTime

print "AES_256_ECB decode:",
startTime = time.time()
de_AES_ECB(key,encrypted)
print time.time()-startTime
#======================================================================
\end{python}
\newpage
\subsection{AES256-CBC}
\begin{python}
# PyCrypto_AES256_CBC
#======================================================================
from Crypto.Cipher import AES
import os
import time
#======================================================================
def start():
    global key
    key = os.urandom(32)                    #AES 256 bits = 32 bytes
    file = open("random.txt","r")
    global text 
    text = file.read()                      #text will be the plaintext
    file.close()
    global iv
    iv = os.urandom(16)                     #block size is 16 bytes
#======================================================================
def en_AES_CBC(key,text,iv):
    padString = ''
    temp = 16-len(text)%16            #counting the padding number
    for i in range(0,temp):           #if surplus 7, then add 9 bytes 9
        padString = padString+chr(temp)
    text = text+padString

    cipher = AES.new(key,AES.MODE_CBC,iv)
    global encrypted 
    encrypted = cipher.encrypt(text)                           #encrypt
#======================================================================
def de_AES_CBC(key,encrypted,iv):
    cipher = AES.new(key,AES.MODE_CBC,iv)
    decrypted = cipher.decrypt(encrypted)                      #decrypt
    temp = len(decrypted)                                    #unpadding
    temp = temp-int(decrypted[-1].encode('hex'),16)
    decrypted = decrypted[:temp]
#======================================================================
start()
#AES256CBC
print "AES_256_CBC encode:",
startTime = time.time()
en_AES_CBC(key,text,iv)
print time.time()-startTime

print "AES_256_CBC decode:",
startTime = time.time()
de_AES_CBC(key,encrypted,iv)
print time.time()-startTime
#======================================================================
\end{python}
\newpage
\subsection{AES256-CTR}
\begin{python}
# PyCrypto_AES256_CTR
#======================================================================
from Crypto.Cipher import AES
from Crypto.Util import Counter
import os
import time
#======================================================================
def start():
    global key
    key = os.urandom(32)                    #AES 256 bits = 32 bytes
    file = open("random.txt","r")
    global text 
    text = file.read()                      #text will be the plaintext
    file.close()
#======================================================================
def en_AES_CTR(key,text):
    padString = ''
    temp = 16-len(text)%16            #counting the padding number
    for i in range(0,temp):           #if surplus 7, then add 9 bytes 9
        padString = padString+chr(temp)
    text = text+padString

    ctr = Counter.new(128)
    cipher = AES.new(key,AES.MODE_CTR,counter=ctr)
    global encrypted 
    encrypted = cipher.encrypt(text)                           #encrypt
#======================================================================
def de_AES_CTR(key,encrypted):
    ctr = Counter.new(128)
    cipher = AES.new(key,AES.MODE_CTR,counter=ctr)
    decrypted = cipher.decrypt(encrypted)                      #decrypt
    temp = len(decrypted)                                    #unpadding
    temp = temp-int(decrypted[-1].encode('hex'),16)
    decrypted = decrypted[:temp]
#======================================================================
start()
#AES256CTR
print "AES_256_CTR encode:",
startTime = time.time()
en_AES_CTR(key,text)
print time.time()-startTime

print "AES_256_CTR decode:",
startTime = time.time()
de_AES_CTR(key,encrypted)
print time.time()-startTime
#======================================================================
\end{python}
\newpage
\subsection{RSA2048}
\subsubsection{generate key}
\begin{python}
from Crypto.PublicKey import RSA
key = RSA.generate(2048)
f = open('privateKey.txt','w')
f.write(key.exportKey())
f.close()
f = open('publicKey.txt','w')
f.write(key.publickey().exportKey())
f.close()
print key.exportKey()
print key.publickey().exportKey()
\end{python}
\subsubsection{Encode-Decode}
\begin{python}
# PyCrypto_RSA2048.py
#======================================================================
from Crypto.PublicKey import RSA
from Crypto.Cipher import PKCS1_v1_5
from Crypto.Hash import SHA512
from Crypto import Random
import os
import time
#======================================================================
def en_RSA():
    file = open("publicKey.txt","r")                 #import public key
    key = file.read()
    file.close()
    rsakey = RSA.importKey(key)
    cipher = PKCS1_v1_5.new(rsakey)                  
    addnum = rsakey.size()/8-10  #256-11
    file1 = open("temp.txt","w")
    file = open("random.txt","r")
    while 1:
        text = file.read(addnum)
        if len(text)==0:
            break
        file1.write(cipher.encrypt(text))                      #encrypt
    file1.close()
    file.close()
#======================================================================
def de_RSA():
    file = open("privateKey.txt","r")            #import privateKey key
    key = file.read()
    file.close()
    rsakey =  RSA.importKey(key)
    cipher = PKCS1_v1_5.new(rsakey)
    dsize = SHA512.digest_size
    file = open("temp.txt","r")
    file1 = open("ans.txt","w")
    while 1:
        text = file.read(256)
        if len(text)==0:
            break
        sentinel = Random.new().read(15 + dsize)
        file1.write(cipher.decrypt(text, sentinel))            #decrypt
    file.close()
    file1.close()
#======================================================================
#RSA2048
print "RSA_2048    encode:",
startTime = time.time()
en_RSA()
print time.time()-startTime

print "RSA_2048    decode:",
startTime = time.time()
de_RSA()
print time.time()-startTime
#======================================================================
\end{python}
\subsection{SHA512}
\begin{python}
# PyCrypto_SHA512.py
#======================================================================
from Crypto.Hash import SHA512
import os
import time
#======================================================================
def start():
    file = open("random.txt","r")
    global text 
    text = file.read()
    file.close()
#======================================================================
def en_SHA512(text):
    h = SHA512.new()
    h.update(text)
    encrypted = h.hexdigest()
#======================================================================
start()
#SHA512
print "SHA_512  hsah_func:",
startTime = time.time()
en_SHA512(text)
print time.time()-startTime
#======================================================================
\end{python}
\newpage
\section{Cryptography}
\subsection{AES256-ECB}
\begin{python}
# Cryptography_AES256_ECB
#======================================================================
import os
import time
from cryptography.hazmat.primitives.ciphers import Cipher, algorithms
from cryptography.hazmat.primitives.ciphers import modes
from cryptography.hazmat.backends import default_backend
from cryptography.hazmat.primitives import padding
#======================================================================
def start():
    global key
    key = os.urandom(32)                    #AES 256 bits = 32 bytes
    file = open("random.txt","r")
    global text 
    text = file.read()                      #text will be the plaintext
    file.close()
#======================================================================
def en_AES_ECB(key,text):
    global encrypted
    backend = default_backend()
    padder = padding.PKCS7(128).padder()              #set padding mode
    padded_data = padder.update(text)
    padded_data += padder.finalize()                           #padding
    cipher = Cipher(algorithms.AES(key), modes.ECB(), backend=backend)
    encryptor = cipher.encryptor()
    #encrypt
    encrypted = encryptor.update(padded_data) + encryptor.finalize()
#======================================================================
def de_AES_ECB(key,encrypted):
    backend = default_backend()
    cipher = Cipher(algorithms.AES(key), modes.ECB(), backend=backend) 
    decryptor = cipher.decryptor()
    #decrypt
    tdata = decryptor.update(encrypted) + decryptor.finalize()
    unpadder = padding.PKCS7(128).unpadder()    
    data = unpadder.update(tdata)+ unpadder.finalize()       #unpadding
#======================================================================
start()
#AES256ECB
print "AES_256_ECB encode:",
startTime = time.time()
en_AES_ECB(key,text)
print time.time()-startTime

print "AES_256_ECB decode:",
startTime = time.time()
de_AES_ECB(key,encrypted)
print time.time()-startTime
#======================================================================
\end{python}
\newpage
\subsection{AES256-CBC}
\begin{python}
# Cryptography_AES256_CBC
#======================================================================
import os
import time
from cryptography.hazmat.primitives.ciphers import Cipher, algorithms
from cryptography.hazmat.primitives.ciphers import modes
from cryptography.hazmat.backends import default_backend
from cryptography.hazmat.primitives import padding
#======================================================================
def start():
    global key,iv
    key = os.urandom(32)                    #AES 256 bits = 32 bytes
    iv = os.urandom(16)                     #block size is 16 bytes
    file = open("random.txt","r")
    global text 
    text = file.read()                      #text will be the plaintext
    file.close()
#======================================================================
def en_AES_CBC(key,text,iv):
    global encrypted
    backend = default_backend()
    padder = padding.PKCS7(128).padder()              #set padding mode
    padded_data = padder.update(text)
    padded_data += padder.finalize()                           #padding
    cipher = Cipher(algorithms.AES(key), modes.CBC(iv),backend=backend)
    encryptor = cipher.encryptor()
    #encrypt
    encrypted = encryptor.update(padded_data) + encryptor.finalize()
#======================================================================
def de_AES_CBC(key,encrypted,iv):
    backend = default_backend()
    cipher = Cipher(algorithms.AES(key), modes.CBC(iv),backend=backend)
    decryptor = cipher.decryptor()
    #decrypt
    tdata = decryptor.update(encrypted) + decryptor.finalize()
    unpadder = padding.PKCS7(128).unpadder()
    data = unpadder.update(tdata)+ unpadder.finalize()       #unpadding
#======================================================================
start()
#AES256CBC
print "AES_256_CBC encode:",
startTime = time.time()
en_AES_CBC(key,text,iv)
print time.time()-startTime

print "AES_256_CBC decode:",
startTime = time.time()
de_AES_CBC(key,encrypted,iv)
print time.time()-startTime
#======================================================================
\end{python}
\newpage
\subsection{AES256-CTR}
\begin{python}
# Cryptography_AES256_CTR
#======================================================================
import os
import time
from cryptography.hazmat.primitives.ciphers import Cipher, algorithms
from cryptography.hazmat.primitives.ciphers import modes
from cryptography.hazmat.backends import default_backend
from cryptography.hazmat.primitives import padding
#======================================================================
def start():
    global key,ctr
    key = os.urandom(32)                    #AES 256 bits = 32 bytes
    ctr = os.urandom(16)                    #block size is 16 bytes
    file = open("random.txt","r")
    global text 
    text = file.read()                      #text will be the plaintext
    file.close()
#======================================================================
def en_AES_CTR(key,text,ctr):
    global encrypted
    backend = default_backend()
    padder = padding.PKCS7(128).padder()              #set padding mode
    padded_data = padder.update(text)
    padded_data += padder.finalize()                           #padding
    cipher = Cipher(algorithms.AES(key),modes.CTR(ctr),backend=backend)
    encryptor = cipher.encryptor()
    #encryptor
    encrypted = encryptor.update(padded_data) + encryptor.finalize()
#======================================================================
def de_AES_CTR(key,encrypted,ctr):
    backend = default_backend()
    cipher = Cipher(algorithms.AES(key),modes.CTR(ctr),backend=backend)
    decryptor = cipher.decryptor()
    #decrypt
    tdata = decryptor.update(encrypted) + decryptor.finalize()
    unpadder = padding.PKCS7(128).unpadder()
    data = unpadder.update(tdata)+ unpadder.finalize()       #unpadding
#======================================================================
start()
#AES256CTR
print "AES_256_CTR encode:",
startTime = time.time()
en_AES_CTR(key,text,ctr)
print time.time()-startTime

print "AES_256_CTR decode:",
startTime = time.time()
de_AES_CTR(key,encrypted,ctr)
print time.time()-startTime
#======================================================================
\end{python}
\newpage
\subsection{RSA2048}
\subsubsection{generate key}
\begin{python}
from cryptography.hazmat.backends import default_backend
from cryptography.hazmat.primitives.asymmetric import rsa
from cryptography.hazmat.primitives import serialization

private_key = rsa.generate_private_key(
    public_exponent=65537,key_size=2048,backend=default_backend())
pem = private_key.private_bytes(
    encoding=serialization.Encoding.PEM,
    format=serialization.PrivateFormat.TraditionalOpenSSL,
    encryption_algorithm=serialization.NoEncryption()
)
f = open('privateKey.txt','w')
f.write(pem)
f.close()
pem = private_key.public_key().public_bytes(
    encoding=serialization.Encoding.PEM,
    format=serialization.PublicFormat.SubjectPublicKeyInfo
)
f = open('publicKey.txt','w')
f.write(pem)
f.close()
\end{python}
\subsubsection{Encode-Decode}
\begin{python}
# Cryptography_RSA2048
#======================================================================
import os
import time
from cryptography.hazmat.primitives.asymmetric import padding
from cryptography.hazmat.backends import default_backend
from cryptography.hazmat.primitives.asymmetric import rsa
from cryptography.hazmat.primitives import serialization
#======================================================================
def en_RSA():
    def en_RSA():
    file = open("publicKey.txt","r")
    public_key = serialization.load_pem_public_key(
        file.read(),
        backend=default_backend()
    )
    file.close()
    addnum = 245
    file1 = open("temp.txt","w")
    file = open("random.txt","r")
    while 1:
        text = file.read(addnum)
        if len(text)==0:
            break
        file1.write(public_key.encrypt(text,apadding.PKCS1v15()))
    file1.close()
    file.close()
#======================================================================
def de_RSA():
    file = open("privateKey.txt","r")
    private_key = serialization.load_pem_private_key(
        file.read(),
        password=None,
        backend=default_backend()
    )
    file.close()
    file = open("temp.txt","r")
    file1 = open("ans.txt","w")
    while 1:
        text = file.read(256)
        if len(text)==0:
            break
        file1.write(private_key.decrypt(text,apadding.PKCS1v15()))
    file.close()
    file1.close()
#======================================================================
#RSA2048
print "RSA_2048    encode:",
startTime = time.time()
en_RSA()
print time.time()-startTime

print "RSA_2048    decode:",
startTime = time.time()
de_RSA()
print time.time()-startTime
#======================================================================
\end{python}
\newpage
\subsection{SHA512}
\begin{python}
# Cryptography_SHA512
#======================================================================
import os
import time
from cryptography.hazmat.primitives import hashes
from cryptography.hazmat.backends import default_backend
#======================================================================
def start():
    file = open("random.txt","r")
    global text 
    text = file.read()
    file.close()
#======================================================================
def en_SHA_512(text):
    digest = hashes.Hash(hashes.SHA512(), backend=default_backend())
    digest.update(text)
    ans = digest.finalize().encode('hex')
#======================================================================
start()
#SHA512
print "SHA_2_512   encode:",
startTime = time.time()
en_SHA_512(text)
print time.time()-startTime
#======================================================================
\end{python}
\newpage
\section{比較}
\begin{table}[H]
\centering
\caption{執行時間(單位:秒)}
\label{my-label}
\begin{tabular}{|l|l|l|}
\hline
\textbf{}                 & \textbf{PyCrypto}                     & \textbf{Cryptography}                 \\ \hline
\textbf{Encode AES256EBC} & 4.91898                               & 1.47442                               \\ \hline
\textbf{Decode AES256EBC} & 5.09097                               & 1.43221                               \\ \hline
\textbf{Encode AES256CBC} & 5.48372                               & 2.39367                               \\ \hline
\textbf{Decode AES256CBC} & 5.33132                               & 1.44257                               \\ \hline
\textbf{Encode AES256CTR} & 6.44351                               & 1.45824                               \\ \hline
\textbf{Decode AES256CTR} & 6.42914                               & 1.46113                               \\ \hline
\textbf{Encode RSA2048}   & {\color[HTML]{CB0000} 1608.21340}     & {\color[HTML]{CB0000} 79.65453}       \\ \hline
\textbf{Decode RSA2048}   & {\color[HTML]{CB0000} 24,809.77850}   & {\color[HTML]{CB0000} 1673.72978}     \\ \hline
\textbf{Encode SHA512}    & 1.71591                               & 0.87150                               \\ \hline
\end{tabular}
\end{table}
\quad 由上表可知:
\begin{enumerate}
    \item Cryptography的執行效率高於PyCrypto
    \item Hash function 速度比加密快很多
	\item AES除了在Cryptography的CBC中加解密速度不一致,其餘加解密時間差不多
	\item 由於PyCrypto的AES CTR時間在AES中最久,因此推測PyCrypto AES CTR並未作分散式處理
	\item \color[HTML]{CB0000}RSA運算十分費時\color[HTML]{000000}
	\item RSA解密時間會比加密來的\color[HTML]{CB0000}大許多
\end{enumerate}
\end{document}
